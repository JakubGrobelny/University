\documentclass[12pt, a4paper, oneside]{report}

\usepackage[utf8]{inputenc}
\usepackage[T1]{fontenc}
\usepackage[polish]{babel}

\title{\Huge Funkcyjny projekt programistyczny\\
       \Large Kompilator + maszyna wirtualna}
       \author{Jakub Grobelny\\ Łukasz Deptuch\\ Kamil Michalak}
\date{}

\begin{document}

\begin{titlepage}
    \maketitle
    \thispagestyle{empty}
\end{titlepage}

\section*{Krótki opis projektu}
Naszym celem jest zaprojektowanie nowego języka programowania i napisanie 
jego kompilatora w Haskellu. Kompilator generował będzie \textit{bytecode} dla
naszej własnej maszyny wirtualnej, która zostanie napisana w języku C.

\section*{Cechy języka}
\begin{itemize}
    \item Statycznie typowany język funkcyjny z ewaluacją gorliwą.
    \item System typów oparty na systemie Hindleya-Milnera, rozszerzony o 
          polimorfizm \textit{ad hoc} w postaci klas typów inspirowanych 
          tymi z Haskella.
    \begin{itemize}
        \item inferencja typów
        \item polimorfizm parametryczny
        \item klasy typów
        \item algebraiczne typy danych
        \item krotki i rekordy
    \end{itemize}
    \item Kompilacja do kodu pośredniego wykonywanego przez zewnętrzne 
          środowisko uruchomieniowe. Pozwala to na większą przenośność
          skompilowanego kodu.
    \item Automatyczne odśmiecanie pamięci.
    \item Gwarancja optymalizacji rekursji ogonowej.
    \item Możliwość programowania imperatywnego w stylu języka \textit{ML}.
    \item Możliwość definiowania własnych operatorów wraz z priorytetami i 
          łącznością.
    \item System importowania modułów podobny do Haskella.
\end{itemize}

\section*{Zadania do wykonania}
\begin{enumerate}

    \item Wymyślenie języka.
    \begin{itemize}
        \item Zaprojektowanie i opisanie systemu typów i semantyki języka.
        \item Wymyślenie i formalne opisanie składni.
        \item Zaprojektowanie systemu importowania modułów.
    \end{itemize}

    \item Napisanie kompilatora w Haskellu.
    \begin{itemize}
        \item Ustalenie składni abstrakcyjnej dla reprezentacji języka.
        \item Napisanie parsera ustalonej składni konkretnej.
        \item Implementacja importowania modułów.
        \item Wstępna implementacja systemu typów.
        \item Napisanie prostego interpretera i \textit{REPL} do debugowania.
        \item Dodanie klas typów do systemu typów.
        \item Generowanie \textit{bytecode}'u maszyny wirtualnej.
        \item Dodanie optymalizacji kodu wynikowego.
    \end{itemize}

    \item Napisanie maszyny wirtualnej w C.
    \begin{itemize}
        \item Wymyślenie odpowiedniego zestawu instrukcji.
        \item Zaprojektowanie formatu plików obsługiwanego przez maszynę 
              wirtualną.
        \item Napisanie interpretera instrukcji.
        \item Implementacja odśmiecania pamięci.
        \item Zaprojektowanie i zaimplementowanie 
              \textit{Foreign Function Interface} w celu umożliwienia
              wywoływania funkcji napisanych w C.
    \end{itemize}

    \item Dalsze rozwijanie języka oraz kompilatora.
    \begin{itemize}
        \item Rozszerzenie systemu modułów o importowanie skompilowanych
              plików zamiast jedynie kodu źródłowego.
        \item Automatyczne wyprowadzanie instancji pewnych klas typów przez
              kompilator.
        \item Napisanie prawdziwego \textit{REPL}.
        \item Dodawanie różnych rozszerzeń systemu typów zaczerpniętych z 
              Haskella.
        \item Polepszanie jakości błędów kompilacji.
    \end{itemize}

    \item Dalsze rozwiajnie maszyny wirtualnej.
    \begin{itemize}
        \item Współbieżność.
    \end{itemize}

\end{enumerate}

\end{document}