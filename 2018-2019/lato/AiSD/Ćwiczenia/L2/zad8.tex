\documentclass[12pt]{article}
\usepackage[utf8]{inputenc}
\usepackage[T1]{fontenc}
\usepackage[polish]{babel}
\usepackage{amsmath}
\usepackage{amsthm}
\usepackage[margin=0.7in]{geometry}
\usepackage{dsfont}
\usepackage{mdframed}
\usepackage[bottom]{footmisc}
\usepackage{abstract}
\usepackage{titlesec}
\usepackage{gensymb}
\usepackage{algorithmicx}
\usepackage{algpseudocode}

\newtheorem*{theorem*}{Twierdzenie}
\newtheorem{definition}{Definicja}
\newtheorem*{assumption*}{Założenie}
\newtheorem{lemma}{Lemat}
\newtheorem{fact}{Fakt}

\titleformat{\section}[block]{\Large\bfseries\filcenter}{}{0em}{}

\title{\bfseries Algorytmy i Struktury Danych\\\Large Lista 2, Zadanie 8}
\date{}
\author{\large Jakub Grobelny}

\begin{document}
\begin{titlepage}
\maketitle
\thispagestyle{empty}

\section{Treść zadania}

8. (2pkt) Ułóż algorytm, który dla danych liczb naturalnych $a$ i $b$, sprawdza,
czy zachłanna strategia dla problemu wydawania reszty jest poprawna, gdy zbiór
nominałów jest równy $X = \{1, a, b\}$.

\section{Rozwiązanie}

\begin{assumption*}
\normalfont
Załóżmy, bez utraty ogólności, że $1 < a < b$.
\end{assumption*}

\begin{definition}
\normalfont
$G(x)$ oznacza liczbę monet w reprezentacji liczby $x$ dla nominałów $\{1, a, b\}$ 
będącej wynikiem algorytmu zachłannego, $M(x)$ zaś liczbę monet w optymalnej
reprezentacji liczby $x$.
\end{definition}

\begin{fact}\normalfont
$\forall_{x\in\mathds{N}}\, G(x) \geq M(x)$.
\end{fact}

\begin{definition}
\normalfont
Kontrprzykładem nazywamy taki $x \in \mathds{N}$, że $G(x) > M(x)$.
\end{definition}
    
\begin{definition}
\normalfont
Mówimy, że zachłanna strategia dla problemu wydawania reszty jest niepoprawna dla
jakiegoś zbioru nominałów, jeżeli istnieje kontrprzykład.
\end{definition}


\begin{definition}
\normalfont
Notacja $x = (i, j, k)$ dla zbioru nominałów $\{1, a, b\}$ oznacza, że reszta $x$ została wydana przy użyciu $i$
monet nominału 1, $j$ monet nominału $a$ oraz $k$ monet nominału $b$.
\end{definition}

\begin{lemma}\label{lm0}\normalfont
Dla każdego $x$ oraz dowolnej monety $y \in \{1, a, b\}$ zachodzi nierówność
$$M(x) \leq M(x-y) + 1$$
\begin{proof}$ $\\
W oczywisty sposób nierówność zachodzi, gdyż z optymalnej reprezentacji
$x-y$ możemy otrzymać reprezentację $x$ o liczbie monet $M(x-y)+1$ dodając
jedną monetę $y$. W najlepszym przypadku otrzymamy w ten sposób reprezentację
optymalną $x$, w każdym innym reprezentację o większej liczbie monet.
\end{proof}
\end{lemma}

\begin{lemma}\label{lm01}\normalfont
Dla dowolnego $x$ i dowolnego $y \in \{1, a, b\}$ równość
$M(x) = M(x-y) + 1$ zachodzi wtedy, i tylko wtedy, gdy
istnieje optymalna reprezentacja $x$ używająca monety $y$.

\begin{proof}$ $\\
$\Rightarrow$\\
Załóżmy, że $M(x) = M(x-y) + 1$. Wówczas reprezentacja optymalna $x$ mogła
zostać uzyskana poprzez dodanie monety $y$ do optymalnej reprezentacji $x-y$,
czyli istnieje optymalna reprezentacja $x$ używająca monety $y$.\\
$\Leftarrow$\\
Założmy, że istnieje optymalna reprezentacja $x$ używająca monety $y$. Wówczas
możemy zabrać jedną monetę $y$ i otrzymamy reprezentację $x-y$ o liczbie monet
$M(x) - 1$. Z lematu \ref{lm0} wynika, że ta reprezentacja $x-y$ jest optymalna,
więc $M(x) = M(x-y) + 1$.
\end{proof}
\end{lemma}

\begin{lemma}\label{lm1}
\normalfont
Jeżeli istnieje kontrprzykład $x$ dla nominałów $\{1, a, b\}$, to najmniejszy
taki $x$ spełnia nierówność $b + 1 < x < b + a$.

\begin{proof}$ $\newline
Załóżmy, że dla nominałów $\{1, a, b\}$ strategia zachłanna jest niepoprawna.\\\\
1) Ograniczenie dolne:\\    
Niech $x$ będzie kontrprzykładem. Rozpatrzmy przypadki:\\
$1\degree$. $x < b$. Wówczas $M(x) = G(x)$, gdyż mamy do
wyboru jedynie monety $\{1, a\}$. $x$ nie może być więc kontrprzykładem.\\
$2\degree$. $x = b$. Wówczas $M(x) = 1 = G(x)$, czyli tak jak w przypadku pierwszym.\\
$3\degree$. $x = b + 1 $. Wówczas $M(x) = G(x)$, gdyż zachłanna reprezentacja jest
identyczna jak optymalna, czyli $(1, 0, 1)$.\\
Widać w takim razie, iż każdy kontrprzykład musi być większy niż $b + 1$.\\\\
2) Ograniczenie górne:\\
Weźmy dowolny $x \geq b + a$. Załóżmy, że $\forall_{y<x}\,G(y) = M(y)$.\\
Rozpatrzmy przypadki:\\
$1\degree$. Moneta o nominale $b$ jest w reprezentacji optymalnej.
Wówczas
$$G(x) = G(x-b) + 1 \stackrel{\text{zał.}}{=} M(x-b) + 1 \stackrel{\text{lem.\ref{lm01}}}{=} M(x)$$
$2\degree$. Moneta o nominale $a$ jest w reprezentacji optymalnej.\\
$$G(x) = G(x-b) + 1 \stackrel{\text{zał.}}{=} M(x-b) + 1 
\stackrel{\text{lem.\ref{lm0}}}{\leq} M(x-b-a) + 2 \leq G(x-b-a) + 2=$$
$$= G(x-a) + 1 \stackrel{\text{zał.}}{=} M(x-a) + 1 \stackrel{\text{lem.\ref{lm01}}}{=} M(x) \leq G(x),$$
z czego wynika, że $G(x) = M(x)$.\\
$3\degree$. Moneta o nominale 1 jest w reprezentacji optymalnej.\\
$$G(x) = G(x-b) + 1 \stackrel{\text{zał.}}{=} M(x-b)+1 \stackrel{\text{lem.\ref{lm0}}}{\leq} M(x-b-1)+2 = $$
$$ \leq G(x-b-1) + 2 = G(x-1) + 1 \stackrel{\text{zał.}}{=} M(x-1) + 1 \stackrel{\text{lem.\ref{lm01}}}{=} $$
$$ \stackrel{\text{lem.\ref{lm01}}}{=} M(x) \leq G(x),$$
z czego wynika, że $M(x) = G(x)$.

Pokazaliśmy, że $G(x) = M(x)$ dla dowolnego $x \geq b + a$ jeżeli nie istnieje żaden kontrprzykład mniejszy
niż $b + a$, więc w takim razie musi istnieć taki kontrprzykład mniejszy od
$b + a$ jeżeli strategia zachłanna ma być niepoprawna.\\

Z 1) i 2) mamy, że każdy kontrprzykład jest większy niż $b + 1$ oraz istnieją
kontrprzykłady mniejsze niż $b + a$, więc z tego wynika, iż najmniejszy
kontrprzykład $x$ spełnia nierówność $b + 1 < x < b + a$.
\end{proof}
\end{lemma}


\begin{theorem*}\normalfont
Strategia zachłanna dla problemu wydawania reszty dla zbioru nominałów 
$\{1, a, b\}$ jest niepoprawna wtedy, i tylko wtedy gdy $0 < r < a - q$,
gdzie $b = qa + r$ (czyli $q = \lfloor \frac{b}{a} \rfloor$ oraz $r = b \bmod a$).

\begin{proof}$ $\newline
$\Rightarrow$\\
Niech $b = qa + r$. Załóżmy, że $0 < r < a - q$. Wówczas istnieje
kontrprzykład $x = b + a - 1$, dla którego reprezentacja zachłanna to $(a - 1, 0, 1)$.
Można przedstawić $x$ również jako $(r - 1, q + 1, 0)$, co daje $r-1+q+1=r+q$ monet,
podczas gdy $G(x) = a - 1 + 1 = a$.\\
Z założenia mamy $0 < r + q < a$, więc $G(x) > r+q$, czyli zgodnie z definicją
strategia zachłanna jest niepoprawna.\\\\
$\Leftarrow$\\
Załóżmy, że strategia zachłanna wydawania reszty jest niepoprawna dla nominałów
$\{1, a, b\}$. Wówczas z lematu \ref{lm1} wiemy, że najmniejszy
kontrprzykład $x$ spełnia nierówność $b + 1 < x < a + b$. W takim razie jego 
reprezentacja zachłanna to $(i, 0, 1)$, gdzie $0 < i < a$. Reprezentacja optymalna
to $(j, k, 0)$, gdzie $k > 0$. Zauważmy, że w oczywisty sposób $j < i$. 
W takim razie jeżeli $x$ jest minimalny, to $j = 0$, gdyż w przeciwnym 
razie $x - j$ byłby mniejszym kontrprzykładem.
Mamy więc
$$x = b + i = ka$$
$$b = ka - i = ka - i - a + a = (k-1)a + (a-i)$$
W takim razie $q = k - 1$ oraz $r = a - i$.
Z nierówności $b + 1 < x < a + b$ mamy
$$0 < (a + b) - x = (a + b) - (b + i) = a - i.$$
$i \neq 0$, więc z powyższej nierówności mamy $0< a - i < a$.
Skoro $x$ jest kontrprzykładem, to mamy również nierówność 
$M(x) = k < 1 + i = G(x)$.\\

Ostatecznie mamy więc
$$0 < a - i < a - (k - 1)$$
$$0 < r < a - q$$
\end{proof}

\end{theorem*}

Na podstawie powyższego twierdzenia możemy już łatwo ułożyć algorytm sprawdzający,
czy dla nominałów $\{1, a, b\}$ strategia zachłanna jest poprawna.

\begin{algorithmic}
    \State $r \gets b \mod a$
    \State $q \gets \lfloor \frac{b}{a} \rfloor$
    \If {$0 < r$ \texttt{and} $r < a - q$}
        \Return \texttt{True}
    \Else $\,$
        \Return \texttt{False}
    \EndIf
\end{algorithmic}

\end{titlepage}

\end{document}



