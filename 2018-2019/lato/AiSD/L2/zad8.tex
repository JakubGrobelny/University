\documentclass[12pt]{article}
\usepackage[utf8]{inputenc}
\usepackage[T1]{fontenc}
\usepackage[polish]{babel}
\usepackage{amsmath}
\usepackage{amsthm}
\usepackage[margin=0.7in]{geometry}
\usepackage{dsfont}
\usepackage{mdframed}
\usepackage[bottom]{footmisc}
\usepackage{abstract}
\usepackage{titlesec}

\newtheorem*{theorem*}{Twierdzenie}
\newtheorem*{definition*}{Definicja}


\titleformat{\section}[block]{\Large\bfseries\filcenter}{}{0em}{}

\title{\bfseries Algorytmy i Struktury Danych\\\Large Lista 2, Zadanie 8}
\date{}
\author{\large Jakub Grobelny}

\begin{document}
\begin{titlepage}
\maketitle
\thispagestyle{empty}

\section{Treść zadania}

8. (2pkt) Ułóż algorytm, który dla danych liczb naturalnych $a$ i $b$, sprawdza,
czy zachłanna strategia dla problemu wydawania reszty jest poprawna, gdy zbiór
nominałów jest równy $X = \{1, a, b\}$.

\section{Rozwiązanie}

\begin{definition*}
\normalfont
$G(x)$ oznacza liczbę monet w reprezentacji liczby $x$ dla nominałów $\{1, a, b\}$ 
będącej wynikiem algorytmu zachłannego, $M(x)$ zaś liczbę monet w optymalnej
reprezentacji liczby $x$.
\end{definition*}

\begin{definition*}
\normalfont
Kontrprzykładem nazywamy taki $x \in \mathds{Z}$, że $G(x) > M(x)$.
\end{definition*}
    
\begin{definition*}
\normalfont
Mówimy, że zachłanna strategia dla problemu wydawania reszty jest niepoprawna dla
jakiegoś zbioru nominałów, jeżeli istnieje kontrprzykład.
\end{definition*}


\begin{definition*}
\normalfont
Notacja $x = (i, j, k)$ dla zbioru nominałów $\{1, a, b\}$ oznacza, że reszta $x$ została wydana przy użyciu $i$
monet nominału 1, $j$ monet nominału $a$ oraz $k$ monet nominału $b$.
\end{definition*}

\begin{theorem*}\normalfont
Strategia zachłanna dla problemu wydawania reszty dla zbioru nominałów 
$\{1, a, b\}$ jest niepoprawna wtedy, i tylko wtedy gdy $0 < r < a - q$,
gdzie $b = qa + r$.
\end{theorem*}


\begin{proof}$ $\newline
1) Dowód implikacji w prawą stronę:\\
Niech $b = qa + r$. Załóżmy, że $0 < r < a - q$. Wówczas istnieje
kontrprzykład $x = b + a - 1$, dla którego reprezentacja optymalna to
$(r - 1, q + 1, 0)$, zaś zachłanna $(a - 1, 0, 1)$.
$$M(x) = r - 1 + q + 1 = r + q$$
$$G(x) = a - 1 + 1 = a$$
Z założenia mamy $0 < r + q < a$, więc $G(x) > M(x)$.\\\\
2) Dowód implikacji w lewą stronę:\\
%TODO:%
\end{proof}

\end{titlepage}

\end{document}



