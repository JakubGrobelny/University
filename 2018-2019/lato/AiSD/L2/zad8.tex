\documentclass[12pt]{article}
\usepackage[utf8]{inputenc}
\usepackage[T1]{fontenc}
\usepackage[polish]{babel}
\usepackage{amsmath}
\usepackage{amsthm}
\usepackage[margin=0.7in]{geometry}
\usepackage{dsfont}
\usepackage{mdframed}
\usepackage[bottom]{footmisc}
\usepackage{abstract}
\usepackage{titlesec}

\newtheorem*{theorem*}{Twierdzenie}
\newtheorem*{definition*}{Definicja}


\titleformat{\section}[block]{\Large\bfseries\filcenter}{}{0em}{}

\title{\bfseries Algorytmy i Struktury Danych\\\Large Lista 2, Zadanie 8}
\date{}
\author{\large Jakub Grobelny}

\begin{document}
\begin{titlepage}
\maketitle
\thispagestyle{empty}

\section{Treść zadania}

8. (2pkt) Ułóż algorytm, który dla danych liczb naturalnych $a$ i $b$, sprawdza,
czy zachłanna strategia dla problemu wydawania reszty jest poprawna, gdy zbiór
nominałów jest równy $X = \{1, a, b\}$.

\section{Rozwiązanie}

\begin{definition*}
\theoremstyle{remark}\normalfont
Niech $G(x)$ i $M(x)$ oznaczają kolejno liczbę monet w reprezentacji liczby $x$
będącej wynikiem algorytmu zachłannego oraz liczbę monet w optymalnej
reprezentacji liczby $x$. Wówczas będziemy mówić, że zachłanna strategia
dla problemu wydawania reszty jest niepoprawna, jeżeli $\exists_x\,G(x) > M(x)$.
\end{definition*}

\begin{theorem*}\normalfont
Strategia zachłanna dla problemu wydawania reszty dla zbioru nominałów 
$\{1, a, b\}$ jest niepoprawna wtedy i tylko wtedy gdy $0 < r < a - q$,
gdzie $b = qa + r$.
\end{theorem*}

\begin{proof}
\end{proof}

\end{titlepage}

\end{document}



